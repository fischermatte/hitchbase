%% Erl�uterungen zu den Befehlen erfolgen unter
%% diesem Beispiel.                
\documentclass{article}
\usepackage[ansinew]{inputenc}
\usepackage[T1]{fontenc} 
\usepackage[ngerman]{babel}

\title{Ein Testdokument}
\author{Otto Normalverbraucher}
\date{05. Januar 2004}
\begin{document}

\maketitle
\tableofcontents
\section{Einleitung}

Hier kommt die Einleitung. Ihre �berschrift kommt
automatisch in das Inhaltsverzeichnis.

\subsection{Formeln}

\LaTeX{} ist auch ohne Formeln sehr n�tzlich und
einfach zu verwenden. Grafiken, Tabellen,
Querverweise aller Art, Literatur- und
Stichwortverzeichnis sind kein Problem.

Formeln sind etwas schwieriger, dennoch hier ein
einfaches Beispiel.  Zwei von Einsteins
ber�hmtesten Formeln lauten:
\begin{eqnarray}
E &=& mc^2                                  \\
m &=& \frac{m_0}{\sqrt{1-\frac{v^2}{c^2}}}
\end{eqnarray}
Aber wer keine Formeln schreibt, braucht sich
damit auch nicht zu besch�ftigen.
\end{document}